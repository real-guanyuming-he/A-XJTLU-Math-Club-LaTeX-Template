\documentclass{article}

\usepackage{mydoc}
\usepackage{xjtlumathstyle}
\usepackage{xjtlumaththm}

\usepackage{listings}
\usepackage{xcolor}

\usepackage{graphicx}

\lstset{language=TeX,
basicstyle=\small,
backgroundcolor=\color{black!20},
breaklines=true,
breakindent=0pt% disable indention for line breaks
}

\usepackage{doc}

\usepackage{hyperref}
\hypersetup{colorlinks=true}

\title{Documentation For Package xjtlumath}
\author{Guanyuming He}
\date{\today}

\newcommand{\mycprt}{Copyright \copyright{} Guanyuming He 2021-\the\year.}

\begin{document}
\pagenumbering{gobble}
\maketitle

\vfill
\section*{Copyright}
\noindent\mycprt{} The document is licensed under the MIT license.

\clearpage
\pagenumbering{roman}
\tableofcontents

\part{General Information}
\pagestyle{headings}
\section{What is this package?}
This \LaTeX{} package was originally intended to be used for the materials dept.~of XJTLU math club. Yet I found that my university lack \LaTeX{} templates for students, therefore I decided to extend it into a template for all users.

The package contains several useful commands and environments for mathematical documents, and redefines some existing styles so that they suit the university's style.

\section{History}
At the year 2020, I joined the materials dept.~of XJTLU math club. This department makes materials about math and distribute them to students to help them. At that time, the materials were being prepared with Microsoft Word and MathType. I saw that it was a good opportunity to enhance the materials with the power of \LaTeX, so I proposed to prepare the documents using \LaTeX.

The proposition was passed and our team started to transfer our working environment. During this process, I wrote a tiny package, a predecessor of this package, which was used in our works. Many people in our department knew nearly nothing about \LaTeX, and as I promised that the new preparation process would not be very hard, I managed to define the overall procedure, shared some necessary knowledge about \LaTeX, and taught them how to use the package.

I admit that I was a bit irresponsible at that time. Truly I described how the working procedure was and how should my package be used, but I shared few about \LaTeX. My package and workflow simplified some concepts so that the team didn't need to care about some parts of \LaTeX{} such as the documentclass and preamble, though \LaTeX{} itself is still a lot different from Word. I underestimated the difficulty for my colleagues to learn basic \LaTeX{} typing skills so it was a torment for some of them. I some kind of realized this problem in the later stages, but I was occupied by my own businesses, and didn't pay enough attention to it.

The final results were generally successful and mostly good enough. Yet I feel sorry for my colleagues because this thing was in fact not that easy for them. In addition, I recently took part in the 15\textsuperscript{th} anniversary of XJTLU math club and this somehow strengthened my sense of belonging to the math club. For these reasons, I decided to leave something for XJTLU's students in the future. I rewrote the whole package and added a series of tutorials that covers the basics of \LaTeX{} as well as making materials, so that people in the material dept.~will do much less to make a material in \LaTeX. The tutorials are in the form of a stories, since I think this method adds more fun to the studying.

Apart from that, the general students of XJTLU or even other \LaTeX{} users may find this package and documentation useful. So its final form was decided to be a template with documentation that is published as a GitHub repository under my personal account. This is because the math club it self doesn't have a GitHub account currently. I will transfer this repository to its official account once it has established one.

%\part{Tutorials} inside Tutorial 1
\part{教程}
\pagenumbering{arabic}
\section{教程 1: Ashley的第一份资料}
Ashley 最近加入了资料部门。 现在他被指派写一部分涵盖基本微积分的资料。 他非常兴奋,因为这将是他的第一个资料,也是他对 \LaTeX 的第一次尝试。 首先,他需要知道如何在他的 PC 上使用 \LaTeX{} 和别的宏包。 在本教程中,我们将跟随 Ashley,看看他在资料准备中学到了什么。

\subsection{安装和配置}
要使用 \LaTeX{},Ashley 需要在他的计算机上安装一个 \TeX{} 发行版(可能还有一个编辑器)。 \LaTeX{} 官方网站上列出了几种流行的发行版:\url{https://www.latex-project.org/get/}。 这些发行版的安装和配置非常简单,Ashley 在几分钟内完成了它们。

然后,Ashley 想要一个编辑器来编写 \LaTeX{} 文档。 他了解到 \TeX studio 是一个很好的工作室,所以他下载并安装了它。

Ashley 现在需要做的是获取该宏包的副本并安装它。 (TBD)

\subsection{基本书写}
Ashley,迫不及待地等待开始他的第一项工作,单击打开 Templates 文件夹并导航到 material-book 文件夹。 他打开encapsulation.tex,发现它是这样的:
\begin{lstlisting}
... Some code ...

\input{chapters.tex}

... Some code ...
\end{lstlisting}

正如文件名所暗示的,这是用于制作将作为书籍出版的资料。 一本书包含一些章节,在资料写作中,每一章都是资料主题的一个特定部分。 例如,Ashley 的工作是微积分资料的一部分,因此他需要为他的工作开始一章。 他打开chapter.tex 并发现它是一个空白文档。 然后他添加以下命令开始他的第一章并在 encapsulation.tex 上运行 \LaTeX{} 以检查输出(左侧是他的命令的结果,出现在不同的字体系列中,右侧是 是他的命令,背景为灰色):

\begin{miniexammar}{.45\textandmarginlen}{\fakesectioningdef{1}{Key points in calculus}}
\begin{lstlisting}
\chapter{Key points in calculus}
\end{lstlisting}
\end{miniexammar}
\verb=\chapter{...}= 是一个 \LaTeX{} \emph{command},它以 \verb=\=% 开头
. 用大括号 \verb={}= 包裹的单词形成命令的 \emph{参数}。 在这个地方,参数是章节的标题。 Ashley 注意到 \LaTeX{} 会自动放大标题字体并使它们加粗。 与许多其他排版软件不同的是,用\LaTeX{} 只需要知道要做什么的逻辑思路(例如在某个地方会有一个名为 xxx 的章节),它会为作者控制外观。

Ashley对这个结果很满意。 然后他键入一些段落。 在 \LaTeX{} 中,段落由一个空行分隔。

\begin{miniexammar}{.5\textandmarginlen}{
\fakesectioningdef{1}{Key points in calculus}%
What does Ashley write in these paragraphs? Well, in fact, I don't know. You may find him and ask him yourself.
		
\hspace{1.5em}But, wait a minute, how do I find Ashley when he doesn't really exist? Well, this is a good question.}
\begin{lstlisting}
\chapter{Key points in calculus}
What does Ashley write in these paragraphs? Well, in fact, I don't know. You may find him and ask him yourself.
		
But, wait a minute, how do I find Ashley when he doesn't really exist? Well, this is a good question.
\end{lstlisting}
\end{miniexammar}
Ashley 注意到段落由 \LaTeX{} 自动缩进。 然而,他也注意到该章正下方的段落没有缩进(在中文环境中有所不同)。

除了章节之外,\LaTeX{} 还提供了以下分节命令:
\begin{itemize}
\item section
\item subsection
\item subsubsection
\item paragraph
\item subparagraph
\end{itemize}
您可能会注意到此处包含该paragraph。 事实上,分节命令 \verb=\paragraph= 像其他分节命令一样为段落生成标题。

当 Ashley 转向整个输出时,他看到他的章节出现在目录中:

\begin{miniexammar}{.7\textandmarginlen}{
\faketoc
\fakecontentsline{1}{\textbf{Key points in calculus}}{1}
}
\begin{lstlisting}
\chapter{Key points in calculus}
\end{lstlisting}
\end{miniexammar}
\LaTeX{} 为所有分节命令生成目录\footnote{其深度低于目录 (toc) 深度。 toc的实际生成过程有点复杂,这里就不赘述了}。 出于这个原因,Ashley 必须在 encapsulation.tex 上运行 \LaTeX{} 两次以获得正确的目录。

当 Ashley 为某个部分写了很长的标题时,目录变得非常难看。 为了解决这个问题,Ashley 可以指定目录中使用的节的标题的简短形式,如下所示:
\begin{miniexammar}{.7\textandmarginlen}{
\faketoc
\fakecontentsline{1}{\textbf{A Short Name}}{1}
}
\begin{lstlisting}
\chapter[A Short Name]{A very very long Caption}
\end{lstlisting}
\end{miniexammar}

\subsection{调整字体} \label{subsec:fonts}
到目前为止,Ashley 知道如何指示 \LaTeX{} 做一些基本的事情。 虽然他感觉精力充沛,正在全速写作,但很快就遇到了一些问题。 Ashley 想强调一些关键字,例如“limit”。 他后来了解到命令 \verb=\emph= 指示 \LaTeX{} 强调传递给它的文本,如下所示。

\begin{parexammar}{.38\textandmarginlen}{
Calculus is the study of \emph{limits}.
}
\begin{lstlisting}
Calculus is the study of \emph{limits}.
\end{lstlisting}
\end{parexammar}

这东西挺\LaTeX,因为Ashley 只告诉\LaTeX{} 强调它,而无法控制\LaTeX{} 如何做到这一点。 虽然大多数时候这就足够了,但Ashley想要更多。 他想知道如何明确控制文本的外观,因为 \LaTeX{} 无法满足所有情况下的所有需求。 \LaTeX{} 确实提供了对字体的某些默认操作,Ashley 可以使用它们来控制文本的大小、系列和样式。
\begin{parexammar}{.45\textandmarginlen}{
Ashley can tell \LaTeX{} to adjust the font size like:
{\tiny very very small} {\scriptsize the size of scripts} {\footnotesize the size of foot notes} {\small small font} {\normalsize just being normal} {\large a bit bigger} {\Large large text} {\LARGE very big} {\huge huge} {\Huge damn huge}
}
\begin{lstlisting}
Ashley can tell \LaTeX{} to adjust the font size like:
{\tiny very very small} {\scriptsize the size of scripts} {\footnotesize the size of foot notes} {\small small font} {\normalsize just being normal} {\large a bit bigger} {\Large large text} {\LARGE very big} {\huge huge} {\HUGE damn huge}
\end{lstlisting}
\end{parexammar}
这一次 Ashley 看到了与命令 \verb=\emph{}= 不同的东西。 此处的文本在大括号内,命令与文本一起在括号内给出。 被一对大括号括起来的东西被称为在 \emph{组} 内。 在组内调用的命令会影响整个组。

除了大小,Ashley 还能够控制字体的样式和系列。 如表 \ref{tab:stdfontcmds} 所示,\LaTeX{} 提供了以下命令来控制字体样式和系列:
\begin{table}[htbp]
\begin{tabular}{lll}
Command & Used in a group & Action\\
\hline
\verb=\textrm{...}= & \verb={\rmfamily...}= & {Text in \textrm{Roman} family} \\
\verb=\textsf{...}= & \verb={\sffamily...}= & {Text in \textsf{sans serif} family} \\
\verb=\texttt{...}= & \verb={\ttfamily...}= & {Text in \texttt{typewriter} family} \vspace{.15cm}\\
\verb=\textmd{...}= & \verb={\mdseries...}= & {Text in \textmd{medium} series} \\
\verb=\textbf{...}= & \verb={\bfseries...}= & {Text in \textbf{bold} series} \vspace{.15cm}\\
\verb=\textup{...}= & \verb={\upshape...}= & {Text in \textup{upright} shape} \\
\verb=\textit{...}= & \verb={\itshape...}= & {Text in \textit{italic} shape} \\
\verb=\textsl{...}= & \verb={\slshape...}= & {Text in \textsl{slanted} shape} \\
\verb=\textsc{...}= & \verb={\scshape...}= & {Text in \textsc{small caps} shape} \vspace{.15cm}\\
\verb=\emph{...}= & \verb={\em...}= & {Text \emph{emphasized}}\vspace{.15cm}\\
\verb=\textnormal{...}= & \verb={\normalfont...}= & {Text in default font}
\end{tabular}
\caption{标准字体更改命令和声明}
\label{tab:stdfontcmds}
\end{table}

这些命令中的许多命令都提供了组内使用版本和普通版本。 Ashley 可以根据自己的需要选择使用哪个版本。

Ashley 还想学习如何更改字体颜色。 他很惊讶这个包文档没有提供这样的描述。 与包作者联系后得知,由于资料是黑白打印的,所以没有涉及这个主题,因此更改颜色几乎没有用。

\subsection{键入数学公式}
这是 Ashley 工作中最激动人心的部分 --- 输入公式!尽管他之前在 \LaTeX{} 方面的经验很少,但他已经了解到 \LaTeX{} 可以生成高质量的数学公式,正如他之前在 Math Stack Exchange (\url{https://math.stackexchange .com/})和知乎(\url{https://www.zhihu.com/})。

由于他之前在这些网站上的经验,他对如何在 \LaTeX{} 中编写公式有所了解。

通常,\LaTeX{} 中的公式分为两种类型:\emph{inline} 和\emph{displayed}。前一种类型的公式被一对美元符号包围:\verb=$...$=,而后一种类型的公式被一对双美元符号包围:\verb=$$...$$ =。这些分隔符是原始的 \TeX{} 分隔符。 \LaTeX{} 分别为内联和显示数学提供了两对分隔符:\verb=\(...\)= 和 \verb=\[...\]=。事实上,应该避免使用 \TeX{} 简写 \verb=$$...$$= 来显示数学,因为它可能会导致 \LaTeX{} 中出现奇怪的问题。

顾名思义,inline公式位于一行文本中,而displayed公式则显示在正文之外。

\begin{miniexammar}{.55\textandmarginlen}%
{
The derivative of a function $f$ can be written as $f'$, or as
\[
\frdt{x}
\]
}
\begin{lstlisting}
The derivative of a function $f$ can be written as $f'$, or as
\[
\frdt{x}
\]
\end{lstlisting}
\end{miniexammar}
在这里,Ashley 使用了 xjtlumath 提供的命令:\verb=\frdt=。 此命令有两个参数,其中第一个参数是可选的,默认值为 $f$。 它们分别代表函数和变量。 当Ashley想要将$g$对$x$的导数排版时,他会写
\verb=\frdt[g]{x}=,给出 $\frdt[g]{x}$。 可选参数被包含在两个方括号内传递给命令。 像 Ashley 一样小心,您可能会注意到,在 inline 模式下,公式会在一定程度上缩小,以便它可以放在一行中。

Ashley对渲染效果相当满意。 然后他写了一些方程式和文本。 但是当他想引用一个方程时,问题就出现了。 他如何处理他想要引用的等式? 在这里,Ashley被介绍了另一种给出显示数学的方法:环境equation。

\begin{miniexammar}{.5\textandmarginlen}%
{
The fundamental theorem of calculus can be expressed in the form of Equation \ref{eq:fundthmcal}
\begin{equation}
\int_a^b f(x) \dx = F(b) - F(a) \label{eq:fundthmcal}
\end{equation}
}
\begin{lstlisting}
The fundamental theorem of calculus can be expressed in the form of Equation \ref{eq:fundthmcal}
\begin{equation}
\int_a^b f(x) \dx = F(b) - F(a) \label{eq:fundthmcal}
\end{equation}
\end{lstlisting}
\end{miniexammar}

Ashley 通过他之前的知识理解了这个等式:下划线 (\verb=_=) 将下标 ($a$) 引入积分符号 $\int$ (\verb=\int=),而插入符号 (\verb=^=) 将上标 ($b$) 引入其中。 命令 \verb=\dx= 由 xjtlumath 提供。 如果Ashley直接输入dx,结果会是这样的$\int f(x) dx$,难看。 \verb=\dx= 优化结果。 注意\verb=\dx= 只能用来表示积分变量,因为它在它前面加了一点空格。 如果Ashley 需要使用其他变量,他需要使用命令\verb=\dd=。 xjtlumath 也为多重积分提供了相同的功能。

\begin{miniexammar}{.6\textandmarginlen}%
{
\[
\int f(t) \dd t,\quad
\iint f(x,y) \dxdy,\quad
\iiint f(x,y,z) \dxdydz
\]
\[
\iint f \drdt,\quad
\iiint f \dzdrdt,\quad
\iiint f \drdtdp
\]
}
\begin{lstlisting}
\[
\int f(t) \dd t,\quad
\iint f(x,y) \dxdy,\quad
\iiint f(x,y,z) \dxdydz
\]
\[
\iint f \drdt,\quad
\iiint f \dzdrdt,\quad
\iiint f \drdtdp
\]
\end{lstlisting}
\end{miniexammar}

然而,Ashley 对 \LaTeX{} 中的交叉引用以及环境equation一无所知。让我们向Ashley解释它们。在\LaTeX{} 中,一个环境由\verb=\begin= 命令开始并以\verb=\end= 命令结束。环境的名称被传递给这对命令。环境equation给出了一个显示的数学方程,它是 \emph{被计数的}。equation末尾的 \verb=\label= 命令捕获计数器以及其他一些信息,例如其位置,并将其存储在由赋予 \verb=\label= 的名称表示的标签中。要使用标签,Ashley 需要 \verb=\ref= 命令,该命令打印计数器\footnote{并另外生成一个可点击的超链接,点击时导航到方程的位置,这是 hyperref ,此模板加载的包的效果。}。

equation 环境给出了一个计数器,而 \verb=\[\]= 没有。在后者使用标签会导致标签被定向到另一个计数器,因此仅当该事物被计数时才使用标签。

\subsection{空白管理}
\LaTeX{} 中的空间管理比仅仅输入空格要复杂一些。 Ashley是个细心的人,他很快发现句子后面的空格比单词之间的空格大一点(你可以放大.pdf文件看到这个)。 \LaTeX{} 决定一个空格作为句子末尾的空格,如果
\begin{enumerate}
\item 终止句子的标点符号后紧跟此空格,并且
\item 如果终止句子的标点符号是句号,则句号前的字母要是小写。
\end{enumerate}
大多数句子都按照上述规则结束,但也有一些例外。 例如,\LaTeX{} 可能将 Mr.~ 作为句子结尾的符号,从而产生错误的间距。 在这种情况下,Ashley 需要通过 \verb=~= 和 \verb=\@= 手动配置 \LaTeX{}。

\begin{parexammar}{.5\textandmarginlen}%
{
In another world, Mr.~Ashley was once loved by Miss Scarlett. In this world, Mr.~Ashley has a PC\@. He loves programming on his PC.
}
\begin{lstlisting}
In another world, Mr.~Ashley was once loved by Miss Scarlett. In this world, Mr.~Ashley has a PC\@. He loves programming on his PC.
\end{lstlisting}
\end{parexammar}
Ashley很高兴学会了如何管理空间。 然而很快他又发现了另一个问题。

\begin{parexammar}{.5\textandmarginlen}%
{
Some commands like \LaTeX seems to eat the space after it.
}
\begin{lstlisting}
Some commands like \LaTeX seems to eat the space after it.
\end{lstlisting}
\end{parexammar}
为了解决这个问题,Ashley 需要在命令的末尾添加一个空组(\verb={}=)。

至于数学公式,事情就变得不一样了。 \LaTeX{} 忽略数学模式下的所有空格,无论是内联的还是显示的。 要添加额外的空格,Ashley 必须使用表 \ref{tab:spacemath} 中显示的命令。
\begin{table}[htbp]
\begin{center}
\begin{tabular}{ll}
Command & Effect (approximately) \\
\hline
\verb=\,= & $\frac{3}{18}$ quad (\showwidth{.166666em}) \vspace{5pt}\\
\verb=\:= & $\frac{4}{18}$ quad (\showwidth{.222222em}) \vspace{5pt}\\
\verb=\;= & $\frac{5}{18}$ quad (\showwidth{.277777em}) \vspace{5pt}\\
\verb=\ = (\verb=\= followed by a space) & a space \vspace{5pt}\\
\verb=\quad= & Width of `M' in current font (\showwidth{1em}) \vspace{5pt}\\
\verb=\qquad= & 2 quad (\showwidth{2em})
\end{tabular}
\end{center}
\caption{空格在数学模式下}
\label{tab:spacemath}
\end{table}

\subsection{列表和其他环境}
现在Ashley已经学会了处理文本,他继续他的写作。很快,他不得不再次停止,因为他正在制定一份列表。 起初,他对列表进行了硬编码,如下所示:

\begin{parexammar}{.4\textandmarginlen}{
1. Something

2. Something

3. Something
}
\begin{lstlisting}
1. Something

2. Something

3. Something
\end{lstlisting}
\end{parexammar}
然而,这个解决方案看起来相当愚蠢。 另外,如果Ashley想要更改号码,他需要手动完成。 他想知道 \LaTeX{} 是否有一些更方便的方法来做到这一点。

幸运的是有。 \LaTeX{} 提供了几种处理列表的环境。 一个例子如下所示。

\begin{miniexammar}{.4\textandmarginlen}{
\begin{enumerate}
\item Something
\item Something
\item Something
\end{enumerate}
}
\begin{lstlisting}
\begin{enumerate}
\item Something
\item Something
\item Something
\end{enumerate}
\end{lstlisting}
\end{miniexammar}

现在Ashley几乎拥有他需要知道的一切。 他对自己写的东西很满意,感到很高兴。 现在对他来说重要的是他想以更花哨的方式展示定理、定义和其他东西,以便他的读者可以专注于这些。

为此,xjtlumath 提供了一些花哨的环境。

\begin{miniexammar}{.6\textandmarginlen}{
\begin{definition}[Absolute Convergence]
An infinite series $\infseries{n}{0}$ is said to be absolutely convergent iff
\[
\sum_{n=0}^\infty |a_n|
\]
converges
\end{definition}
}
\begin{lstlisting}
\begin{definition}[Absolute Convergence]
An infinite series $\infseries{n}{0}$ is said to be absolutely convergent iff
\[
\sum_{n=0}^\infty |a_n|
\]
converges
\end{definition}
\end{lstlisting}
\end{miniexammar}

xjtlumath 加载 amsthm,它提供了proof环境。 当 Ashley 使用这个环境写证明时,他发现在环境的末尾出现了一个 q.e.d.~符号。 
\begin{miniexammar}{.5\textandmarginlen}{
Now we prove the mean value theorem for definite integrals. That is, for a continuous function $f$ that is bounded on $[a,b]$, the definite integral $\int_a^b f(x) \dx = f(c)(b-a)$, where $c \in [a,b]$. 
\begin{proof}
Let $m,M$ be the infimum and supermum of $f([a,b])$, respectively.
Therefore, $m\le f \le M$, and
\[
\int_a^b m \dx \le \int_a^b f(x) \dx \le \int_a^b M \dx
\]
, which gives
\begin{align}
m(b-a) \le &\int_a^b f(x) \dx \le M(b-a) \nonumber\\
m \le &\frac{\int_a^b f(x) \dx}{b-a} \le M \label{eq:meanvalint}
\end{align}

Since that $f$ is continuous, it can reach every value between the infimum and supermum of its range. That is, $\exists c \in [a,b], f(c) = \int_a^b f(x) \dx$. Substitute $f(c)$ back to equation \ref{eq:meanvalint} gives what the theorem states.
\end{proof}
}
\begin{lstlisting}[basicstyle=\footnotesize]
Now we prove the mean value theorem for definite integrals. That is, for a continuous function $f$ that is bounded on $[a,b]$, the definite integral $\int_a^b f(x) \dx = f(c)(b-a)$, where $c \in [a,b]$. 
\begin{proof}
Let $m,M$ be the infimum and supermum of $f([a,b])$, respectively.
Therefore, $m\le f \le M$, and
\[
\int_a^b m \dx \le \int_a^b f(x) \dx \le \int_a^b M \dx
\]
, which gives
\begin{align}
m(b-a) \le &\int_a^b f(x) \dx \le M(b-a) \nonumber\\
m \le &\frac{\int_a^b f(x) \dx}{b-a} \le M \label{eq:meanvalint}
\end{align}

Since that $f$ is continuous, it can reach every value between the infimum and supermum of its range. That is, $\exists c \in [a,b], f(c) = \int_a^b f(x) \dx$. Substitute $f(c)$ back to equation \ref{eq:meanvalint} gives what the theorem states.
\end{proof}
\end{lstlisting}
\end{miniexammar}

Ashley 能够通过使用 proof 提供的 \verb=\qedhere= 命令来控制 q.e.d.~符号出现的位置。 如果这个命令在之前被给出,那么它就不会出现在最后。
\begin{miniexammar}{.5\textandmarginlen}{
\begin{proof}
Some words...
\[
a+b=c \qedhere
\]
Som words...
\end{proof}
}
\begin{lstlisting}
\begin{proof}
Some words...
\[
a+b=c \qedhere
\]
Som words...
\end{proof}
\end{lstlisting}
\end{miniexammar}

除了definition和proof,Ashley 还能够使用theorem, proposition, corollary, lemma, axiom, and example. 除了proof之外,这些环境拥有各自的计数器,Ashley 可以简单地使用标签来引用它们。

不过,Ashley的同事 超 抱怨到当他尝试写中文资料时,这些环境的标题还是英文。为了支持中文,xjtlumath给这些环境(不幸的是,在中文环境下,proof的标题也变成了中文,就没有英文版的了)增加了中文版本,只要在最后加一个c即可使用中文版本。

这些设施极大地帮助了Ashley的资料准备,他很快就会完成他的工作……

\section{教程2: Delilah和复杂的数学公式}
Delilah 正在研究有关线性代数的材料的一部分。 随着工作的进行,她将获得处理 \LaTeX{} 中复杂数学公式的能力,尤其是 xjtlumath 中加载的 ams 包提供的那些方法。 

\subsection{多行公式}
线性方程组是线性代数的基本部分。 当 Delilah 尝试输入一组方程时,她遇到了一个问题。 在预定义的 \verb=\[\]= 和环境equation中,她找不到开始新行的选项。 甚至 \LaTeX{} 的换行选项,如 \verb=\\= 和 \verb=\newline= 在那里也不起作用。 一个方程组当然不应该放在一行,那她现在该怎么办? 后来她了解到环境 \emph{aligned} 旨在允许方程组在多行中对齐:
\begin{miniexammar}{.4\textandmarginlen}{
\[
\begin{aligned}
x+y &= 1\\
x-y &= 2
\end{aligned}
\]
}
\begin{lstlisting}
\[
\begin{aligned}
x+y &= 1\\
x-y &= 2
\end{aligned}
\]
\end{lstlisting}
\end{miniexammar}

这里,与符号 \verb=&= 用在方程要对齐的符号之前。 换行符 \verb=\\= 开始一行新的等式。 请注意,此处不能使用其他换行操作。

Delilah 喜欢这个结果,但她觉得方程式太孤单了。 她认为为他们添加一个大花括号会安慰他们。 \LaTeX{} 支持将分隔符放在一组事物之前和之后的语法。
\begin{miniexammar}{.4\textandmarginlen}{
\[
\left\{
\begin{aligned}
x+y &= 1\\
x-y &= 2
\end{aligned}
\right.
\]
}
\begin{lstlisting}
\[
\left\{
\begin{aligned}
x+y &= 1\\
x-y &= 2
\end{aligned}
\right.
\]
\end{lstlisting}
\end{miniexammar}
\verb=\left= 命令定义要放在左边的东西,而\verb=\right= 命令定义要放在右边的东西。 Delilah 不想把任何东西放在右边,所以写了 \verb=.=表示空。

对于一组不需要对齐的方程,或者对于一个太长而不能放在一行中的单个方程,没有对齐的环境 \emph{gathered} 是更好的选择:
\begin{miniexammar}{.45\textandmarginlen}{
\[
\begin{gathered}
\cos {z} = 1 - \frac{z^2}{2!} + \frac{z^4}{4!} - \frac{z^6}{6!} + \cdots \\
= \sum_{n=0}^\infty {\frac{(-1)^n z^{2n}}{(2n!)}}
\end{gathered}
\]
}
\begin{lstlisting}
\[
\begin{gathered}
\cos {z} = 1 - \frac{z^2}{2!} + \frac{z^4}{4!} - \cdots \\
= \sum_{n=0}^\infty {\frac{(-1)^n z^{2n}}{(2n!)}}
\end{gathered}
\]
\end{lstlisting}
\end{miniexammar}

写了几组方程后,Delilah 想引用其中的一组。 她使用equation环境而不是 \verb=\[\]=,但发现方程是作为一个整体编号的。
\begin{miniexammar}{.4\textandmarginlen}{
\begin{equation}
\begin{aligned}
x+y &= 1\\
x-y &= 2
\end{aligned}
\end{equation}
}
\begin{lstlisting}
\begin{equation}
\begin{aligned}
x+y &= 1\\
x-y &= 2
\end{aligned}
\end{equation}
\end{lstlisting}
\end{miniexammar}
所以她很难在一个组中引用单个方程。 amsmath 为此提供环境 align:

\begin{miniexammar}{.4\textandmarginlen}{
\begin{align}
x+y &= 1\\
x-y &= 2
\end{align}
}
\begin{lstlisting}
\begin{align}
x+y &= 1\\
x-y &= 2
\end{align}
\end{lstlisting}
\end{miniexammar}

如果她不想为单个方程编号,则需要在该行的末尾附加 \verb=\nonumber=。
\begin{miniexammar}{.4\textandmarginlen}{
\begin{align}
x+y &= 1\\
z&=10 \nonumber\\
x-y &= 2
\end{align}
}
\begin{lstlisting}
\begin{align}
x+y &= 1\\
z &= 10 \nonumber\\
x-y &= 2
\end{align}
\end{lstlisting}
\end{miniexammar}

如果没有 'ed' 后缀,\emph{gather} 也是一个独立的环境,可以完成gather的工作。 但是普通版本和“ed”版本之间有一个主要区别。 Delilah 发现不可能在对齐或聚集之前再次放置括号,因为它们不需要但被数学环境包围。 此外,它们的宽度固定为文本的宽度,而它们的“ed”版本可以是任何宽度。如果没有 'ed' 后缀,\emph{gather} 也是一个独立的环境,可以完成聚集的工作。 但是普通版本和“ed”版本之间有一个主要区别。 Delilah 发现不可能在对齐或聚集之前再次放置括号,因为它们不需要但被数学环境包围。 此外,它们的宽度固定为文本的宽度,而它们的“ed”版本可以是任何宽度。

与 equation 环境一样,它们的星号版本默认不给出数字。
\begin{miniexammar}{.4\textandmarginlen}{
\begin{align*}
x+y &= 1\\
x-y &= 2
\end{align*}
}
\begin{lstlisting}
\begin{align*}
x+y &= 1\\
x-y &= 2
\end{align*}
\end{lstlisting}
\end{miniexammar}

Delilah 能够将多组方程放在一个列中,只需在组之间添加\verb=&=号即可。
\begin{miniexammar}{.4\textandmarginlen}{
\begin{align*}
x+y &= 1  & a+b &= 3\\
x-y &= 2  & a-b &= 4
\end{align*}
}
\begin{lstlisting}
\begin{align*}
x+y &= 1  & a+b &= 3\\
x-y &= 2  & a-b &= 4
\end{align*}
\end{lstlisting}
\end{miniexammar}
align 自动调整方程组之间的空间。

\subsection{矩阵}
矩阵对于线性代数至关重要,因为它们表示从一个向量空间到另一个特定基底的线性映射。 此外,系数矩阵和增广矩阵便于操作线性方程。

amsmath 提供了多种输入矩阵的环境。
\begin{miniexammar}{.5\textandmarginlen}{
\[
\begin{bmatrix}
1&2&3&4\\
5&6&7&8\\
9&10&11&12\\
13&14&15&16
\end{bmatrix}
\]
}
\begin{lstlisting}
\[
\begin{bmatrix}
1&2&3&4\\
5&6&7&8\\
9&10&11&12\\
13&14&15&16
\end{bmatrix}
\]
\end{lstlisting}
\end{miniexammar}
环境 pmatrix、Bmatrix、vmatrix 和 Vmatrix 分别产生分隔符 \verb=()=、\verb={}=、\verb=||= 和 \verb=|| ||=。

为了在inline模式下使用矩阵,Delilah 使用环境 smallmatrix,它在 amsmath 中没有 p,b,B,v,V 版本,因为决定分隔符是作者的责任。
\begin{miniexammar}{.4\textandmarginlen}{
The matrix $\left(\begin{smallmatrix} a&b\\c&d \end{smallmatrix}\right)$ is so small and cute!
}
\begin{lstlisting}
The matrix $\left(\begin{smallmatrix} a&b\\c&d \end{smallmatrix}\right)$ is so small and cute!
\end{lstlisting}
\end{miniexammar}

当 Delilah 尝试将分数放入矩阵中时,她发现有些烦人事
。\begin{miniexammar}{.3\textandmarginlen}{
\[
\begin{bmatrix}
1&\frac{1}{2}&\frac{1}{3}\\
1&\frac{1}{4}&\frac{1}{5}\\
\end{bmatrix}
\]
}
\begin{lstlisting}
\[
\begin{bmatrix}
1&\frac{1}{2}&\frac{1}{3}\\
1&\frac{1}{4}&\frac{1}{5}\\
\end{bmatrix}
\]
\end{lstlisting}
\end{miniexammar}
上面和下面的分数非常接近,以至于它们相互接触! 这不是 Delilah 想要的,她很惊讶 \LaTeX{} 没有检测到这一点并做一些事情。 幸运的是,在 amsmath 环境中,允许将可选参数传递给 \verb=\\= 以定义行之间的实际垂直空间。 对于分数,2ex 是一个不错的选择。 此外,分数处于inline模式。 \verb=\dfrac= 命令给出displayed模式的分数。
\begin{miniexammar}{.3\textandmarginlen}{
\[
\begin{bmatrix}
1&\dfrac{1}{2}&\dfrac{1}{3}\\[2ex]
1&\dfrac{1}{4}&\dfrac{1}{5}
\end{bmatrix}
\]
}
\begin{lstlisting}
\[
\begin{bmatrix}
1&\dfrac{1}{2}&\dfrac{1}{3}\\[2ex]
1&\dfrac{1}{4}&\dfrac{1}{5}
\end{bmatrix}
\]
\end{lstlisting}
\end{miniexammar}

有时矩阵太大而无法完全显示。 在这些时候,使用省略号很重要。 当 Delilah 写出矩阵的逆时,她使用省略号。
\begin{miniexammar}{.5\textandmarginlen}{
\[
A^{-1} = \frac{1}{\det A}
\begin{bmatrix}
C_{11} & C_{21} & \cdots & C_{n1} \\
C_{12} & C_{22} & \cdots & C_{n2} \\
\vdots & \vdots & \ddots & \vdots \\
C_{n2} & C_{n2} & \cdots & C_{nn} \\
\end{bmatrix}
\]
}
\begin{lstlisting}
\[
A^{-1} = \frac{1}{\det A}
\begin{bmatrix}
C_{11} & C_{21} & \cdots & C_{n1} \\
C_{12} & C_{22} & \cdots & C_{n2} \\
\vdots & C\vdots & \ddots & \vdots \\
C_{n2} & C_{n2} & \cdots & C_{nn} \\
\end{bmatrix}
\]
\end{lstlisting}
\end{miniexammar}

\subsection{文字和运算符}
要将文本放入数学环境中,Delilah 使用 amsmath 提供的 \verb=\text= 命令。
\begin{miniexammar}{.57\textandmarginlen}{
\begin{definition}[Null Space]
The null space of an $m \times n$ matrix $A$, written as $\Nul A$, is the set of all solutions
of the homogeneous equation $A\vec{x} = \vec{0}$. In set notation,
\[
\Nul A = \{\vec{x}:\vec{x} \text{ is in } \mathbb{R}^n \text{ and } A\vec{x} = \vec{0} \}
\]
\end{definition}
}
\begin{lstlisting}
\begin{definition}[Null Space]
The null space of an $m \times n$ matrix $A$, written as $\Nul A$, is the set of all solutions
of the homogeneous equation $A\vec{x} = \vec{0}$. In set notation,
\[
\Nul A = \{\vec{x}:\vec{x} \text{ is in } \mathbb{R}^n \text{ and } A\vec{x} = \vec{0} \}
\]
\end{definition}
\end{lstlisting}
\end{miniexammar}

\verb=\Nul=, \verb=\sin=, ... 等命令是数学运算符。 \LaTeX{} 中的部分预定义数学运算符显示在表 \ref{tab:predefmathop} 中。
\begin{table}[hbpt]·
\begin{center}
\small
\begin{tabular}{cl|cl|cl}
Result & Command & Result & Command & Result & Command \\
\hline
arccos & \verb=\arccos= & arcsin & \verb=\arcsin= & arctan & \verb=\arctan= \\
cos & \verb=\cos= & sin & \verb=\sin= & tan & \verb=\tan= \\
cot & \verb=\cot= & sec & \verb=\sec= & csc & \verb=\csc= \\
cosh & \verb=\cosh= & sinh & \verb=\sinh= & tanh & \verb=\tanh= \\
lim & \verb=\lim= & lim inf & \verb=\liminf= & lim sup & \verb=\limsup= \\
ln & \verb=\ln= & log & \verb=\log= & lg & \verb=\lg= \\
max & \verb=\max= & min & \verb=\min= & sup & \verb=\sup= \\
inf & \verb=\inf= &  &  &  &  \\
ker & \verb=\ker= & det & \verb=\det= & exp & \verb=\exp= 
\end{tabular}
\end{center}
\caption{一些预定义的数学运算符}
\label{tab:predefmathop}
\end{table}

实际上,运算符\verb=\Nul= 和\verb=\Span= 是由xjtlumath 定义的,如西浦一年级线性代数教科书中的形式。 此外,xjtlumath 更改了 \LaTeX{} 中的默认 \verb=\vec= 命令,以便矢量以粗体形式出现,而不是在其上方带有箭头。

一些运算符,如 \verb=\lim=,旨在支持对它取极限。 也就是说,在displayed模式下,当试图使用 \verb=_= 给这样的运算符一下标时,下标将出现在运算符的底部。
\begin{parexammar}{.4\textandmarginlen}{
\[
\lim_{x\to 0} f(x)
\]
}
\begin{lstlisting}
\[
\lim_{x\to 0} f(x)
\]
\end{lstlisting}
\end{parexammar}

Delilah 能够通过使用 \verb=\limits= 和 \verb=\nolimits= 来明确控制极限样式。 请注意,这两个命令只能在支持采取极限的操作后使用。
\begin{parexammar}{.4\textandmarginlen}{
$\lim\limits_{x \to 0}f(x)$
\[
\lim\nolimits_{x\to 0} f(x)
\]
}
\begin{lstlisting}
$\lim\limits_{x \to 0}f(x)$
\[
\lim\nolimits_{x\to 0} f(x)
\]
\end{lstlisting}
\end{parexammar}

\subsection{限制符}
Delilah 已经知道如何键入基本的分隔符。 对于圆括号和方括号,纯文本就行; 由于大括号是由 \LaTeX{} 保留的,因此 Delilah 需要在每个大括号之前添加一个反斜杠。
\begin{parexammar}{.45\textandmarginlen}{
The range of a function may be expressed explicitly by its domain and itself: the range of $f: X \to Y$ is $f(X)$.

In some context the arguments of a function are enclosed by square brackets: $f[x]$.

Curly brackets are often used to show a set: $S := \{2,4,\cdots\}$.
}
\begin{lstlisting}
The range of a function may be expressed explicitly by its domain and itself: the range of $f: X \to Y$ is $f(X)$.

In some context the arguments of a function are enclosed by square brackets: $f[x]$.

Curly brackets are often used to show a set: $S := \{2,4,\cdots\}$.
\end{lstlisting}
\end{parexammar}

然而,在某些情况下,被包围的表达式具有不同的高度,这时结果变得不令人满意。 \LaTeX{} 提供了一种机制,使用户能够自动或手动调整分隔符的大小。

在一对分隔符周围添加 \verb=\left= 和 \verb=\right= 会自动调整它们的大小以匹配所包含的表达式。 但是,有时我们希望分隔符更大或更小,这就是我们需要手动调整大小的时候。
\begin{parexammar}{.5\textandmarginlen}{
\[
\left( \frac{1}{2} \right) 
\Bigg( \bigg( \Big( \big( x\big) \Big) \bigg) \Bigg)
\]
}
\begin{lstlisting}
\[
\left(\frac{1}{2}\right) \Bigg( \bigg( \Big( \big( x\big) \Big) \bigg) \Bigg)
\]
\end{lstlisting}
\end{parexammar}

Delilah 曾经错配过一对分隔符,但她发现这两个符号的高度还是一样的。 此外,她还了解到点“.” 可用于通知 \LaTeX{} 不插入任何内容。
\begin{parexammar}{.5\textandmarginlen}{
A fraction enclosed left by a parenthesis, and right by a curly bracket.
\[
\left(\frac{a}{b}\right\}
\]
A system of equations
\[
\left\{
\begin{aligned}
x+y&=1\\
x-y&=2
\end{aligned}
\right.
\]
}
\begin{lstlisting}
A fraction enclosed left by a parenthesis, and right by a curly bracket.
\[
\left(\frac{a}{b}\right\}
\]
A system of equations
\[
\left\{
\begin{aligned}
x+y&=1\\
x-y&=2
\end{aligned}
\right.
\]
\end{lstlisting}
\end{parexammar}

现在 Delilah 知道如何正确处理分隔符,但是重复的 \verb=\left= 和 \verb=\right= 确实让她感到恶心。 xjtlumath 通过提供这些预定义的分隔符组来简化工作。以下的例子展示了这些预定义的分隔符组,和对分隔符大小的控制。
\begin{parexammar}{.5\textandmarginlen}{
These delimiters are defined by xjtlumath.
\[
\rbra{x},\ \sbra{x},\ \cbra{x},\ \abs{x},\ \floor{x},\ \ceiling{x}
\]
One can also change the delimiter resizer.
\[
\abs[\bigg]{x}
\]
}
\begin{lstlisting}
These delimiters are defined by xjtlumath.
\[
\rbra{x},\ \sbra{x},\ \cbra{x},\ \abs{x},\ \floor{x},\ \ceiling{x}
\]
One can also change the delimiter resizer.
\[
\abs[\bigg]{x}
\]
\end{lstlisting}
\end{parexammar}

\subsection{符号}
数学环境中的标准 \LaTeX{} 字体整洁干净。 然而在一些特殊场合,Delilah 想改变一些符号的字体。 例如,为了表示一些常规集合,她使用黑板字体。
\begin{parexammar}{.4\textandmarginlen}{
\[
\mathbb{R}\quad \mathbb{N}\quad \mathbb{Q}\quad \mathbb{Z}
\]
}
\begin{lstlisting}
\[
\mathbb{R}\quad \mathbb{N}\quad \mathbb{Q}\quad \mathbb{Z}
\]
\end{lstlisting}
\end{parexammar}

每次都写 \verb=\mathbb= 有点烦人。 为此,xjtlumath 为它们定义了简写。
\begin{parexammar}{.45\textandmarginlen}{
\[
\setr \quad \setq \quad \setz \quad \setn \quad \setnp
\]
}
\begin{lstlisting}
\[
\setr \quad \setq \quad \setz \quad \setn \quad \setnp
\]
\end{lstlisting}
\end{parexammar}

其他字体控制方法如我们在 \ref{subsec:fonts} 小节中讨论过的。 例如,\verb=\mathrm= 给出\textrm{Roman} 家族的字体,而\verb=\mathbf= 给出\textbf{bold} 系列的字体。
\begin{parexammar}{.45\textandmarginlen}{
\[
\mathrm{Like normal text} \quad \mathbf{bold}
\]
}
\begin{lstlisting}
\[
\mathrm{Like normal text} \quad \mathbf{bold}
\]
\end{lstlisting}
\end{parexammar}

\section{教程 3: 月处理floats}
月正在为数学部的月刊准备材料。 她想让她的材料有趣且易于理解,因此她使用了许多图形和表格。

图形、表格和许多其他占据随机宽度和高度(通常很大)区域的东西在 \LaTeX{} 中被视为 \emph{floats}。 Floats在今天的文档中很常见,但它们在排版时会造成很大的麻烦。 在本节中,我们将与月一起了解如何在 \LaTeX{} 中处理floats。

\subsection{插入图像}
要在 \LaTeX{} 中插入图像,包 graphicsx(由模板文件加载)是一个不错的选择。 它提供了接受输入图像文件名和一些可选说明符的命令 \verb=\includegraphics=。
\begin{miniexammar}{.4\textandmarginlen}{
\includegraphics[width=\textwidth]{assets/examplelogo.jpg}
}
\begin{lstlisting}
\includegraphics[width=\textwidth]{assets/examplelogo.jpg}
\end{lstlisting}
\end{miniexammar}

月很快发现,简单地使用这个命令并不是一个好的选择,因为如果图像没有足够的垂直空间,它会被放置在下一页,留下一个很大的空白区域,非常难看。 此外,她无法为图像提供标题或引用它。

所以,月用使图像成为 \emph{figure} 的 figure 环境包裹了图像。
\begin{miniexammar}{.45\textandmarginlen}{
% figure with a caption cannot be placed inside a minipage, so we fake it here. 
\includegraphics[width=\textwidth]{assets/examplelogo.jpg}
\begin{center}
\hypertarget{fakedcaption}{Figure 1: Example Logo}
\end{center}
Figure \hyperlink{fakedcaption}{1} shows the figure Yue uses.
}
\begin{lstlisting}
\begin{figure}
\includegraphics[width=\textwidth]{assets/examplelogo.jpg}
\caption{Example Logo}
\label{fig:example}
\end{figure}
Figure \ref{fig:example} shows the figure Yue uses.
\end{lstlisting}
\end{miniexammar}

\LaTeX{} 会自动给它一个数字,以便月能够引用它。 请注意,由于内部实现原因,\verb=\label= 只能紧跟在 \verb=\caption= 之后,以免引用错误。

\subsection{表}
即使使用数字确实需要额外的环境,但它仍然很简单,月很快就熟悉了。 然而,在 \LaTeX{} 中处理表格更为复杂。

要在 \LaTeX 中生成类似表格的内容,月必须使用特殊环境。 tabular和array是其中的两个具体例子。 事实上,这两种环境在大多数方面是相似的,一个主要区别是array经常用于数学模式。

array 和 tabular 的语法类似于 Delilah 使用的矩阵环境之一,尽管在这里月必须明确指定列行为。
\begin{miniexammar}{.4\textandmarginlen}{
\begin{tabular}{|c|c|}
\hline 
Entry 1 & Entry 2\\
\hline
a & b\\
\hline
\end{tabular}
}
\begin{lstlisting}
\begin{tabular}{|c|c|}
\hline 
Entry 1 & Entry 2\\
\hline
a & b\\
\hline
\end{tabular}
\end{lstlisting}
\end{miniexammar}

月不太明白\verb=|c|c|= 是什么意思,所以她在网上搜索了这个。 这告诉她传递给 tabular 的这个参数指定了每一列。 字母 c 告诉表格该列中的内容应该居中。 另外两个对齐说明符 l 和 r 分别可用于“左”和“右”。 竖线表示在此位置应插入一条垂直线(比较指示在当前行顶部插入水平线的 \verb=\hline= 命令)

当给定 c、l 和 r 时,列的宽度由内容的宽度决定。 通过使用另一个说明符“p”,Yue 能够控制列的宽度,其中内容是左对齐的。
\begin{miniexammar}{.4\textandmarginlen}{
\begin{tabular}{|c|p{2cm}|}
\hline 
Entry 1 & Entry 2\\
\hline
a & b\\
\hline
\end{tabular}
}
\begin{lstlisting}
\begin{tabular}{|c|p{2cm}|}
\hline 
Entry 1 & Entry 2\\
\hline
a & b\\
\hline
\end{tabular}
\end{lstlisting}
\end{miniexammar}

当有许多列具有相同的说明符时,月可以使用这种语法 \verb=*{num}{spe}= 来重复说明符,其中 num 是重复的次数,spe 是说明符。
\begin{miniexammar}{.4\textandmarginlen}{
\begin{tabular}{|*{7}{c|}}
\hline 
a&a&a&a&a&a&a\\
\hline
a&a&a&a&a&a&a\\
\hline
\end{tabular}
}
\begin{lstlisting}
\begin{tabular}{|*{7}{c|}}
\hline 
a&a&a&a&a&a&a\\
\hline
a&a&a&a&a&a&a\\
\hline
\end{tabular}
\end{lstlisting}
\end{miniexammar}

月不喜欢用线来分行列的表格,因为她认为它们不整洁。 她想用空间来分隔内容。 她可以在列说明符之间使用 \verb=@{\hspace{}}= 来指定列间空间,并在 \verb=\\= 之前使用 \verb=\vspace{}= 在下一行之前添加额外的空间。
\begin{miniexammar}{.4\textandmarginlen}{
\begin{tabular}{c@{\hspace{1cm}}cc}
a & b & b\\
a & b & b\vspace{.5cm}\\
c & d & d\vspace{.5cm}\\
\end{tabular}
}
\begin{lstlisting}
\begin{tabular}{c@{\hspace{1cm}}cc}
a & b & b\\
a & b & b\vspace{.5cm}\\
c & d & d\vspace{.5cm}\\
\end{tabular} 
\end{lstlisting}
\end{miniexammar}

实际上,在\verb=@{}=里面,月不仅可以使用空格,还可以使用其他任何内容。
\begin{miniexammar}{.45\textandmarginlen}{
\begin{tabular}{c@{ <POLICE, stay away> }c}
crime scene & people \\
crime scene & people \\
\end{tabular}
}
\begin{lstlisting}
\begin{tabular}{c@{ <POLICE, stay away> }c}
crime scene & people \\
crime scene & people \\
\end{tabular}
\end{lstlisting}
\end{miniexammar}

与环境figure一样,环境table被设计为接受类似表的内容。
\begin{miniexammar}{.4\textandmarginlen}{
\begin{tabular}{cc}
a & b \\
c & d \\
\end{tabular}
\begin{center}
\hypertarget{fakedcaptiontab}{Table 1: Example Table}
\end{center}
Table \hyperlink{fakedcaptiontab}{1} shows a table.
}
\begin{lstlisting}
\begin{table}
\begin{tabular}{cc}
a & b \\
c & d \\
\end{tabular}
\caption{Example Table}
\label{tab:example}
\end{table}
Table \ref{tab:example} shows a table.
\end{lstlisting}
\end{miniexammar}

既然月了解了如何在\LaTeX{} 中操作类似表格的内容,她很快就会继续工作。 在某些时候,她必须创建一个包含至少 20 列的超大表,她发现它超出了页面。 当她尝试使用 \verb=\small= 来减小文本的大小时,她发现她必须为表格的每个条目复制它,这意味着要重复数百次。 为了控制整体风格,Yue 需要在 table 的开头和 tabular 的开头之间给出控制命令。

\begin{miniexammar}{.5\textandmarginlen}{
{
\tiny
\begin{tabular}{|*{14}{c|}}
\hline 
a&a&a&a&a&a&a&a&a&a&a&a&a&a\\
\hline
a&a&a&a&a&a&a&a&a&a&a&a&a&a\\
\hline
\end{tabular}
}
}
\begin{lstlisting}
\begin{table}
\small
\begin{tabular}{|*{14}{c|}}
\hline 
a&a&a&a&a&a&a&a&a&a&a&a&a&a\\
\hline
a&a&a&a&a&a&a&a&a&a&a&a&a&a\\
\hline
\end{tabular}
\end{table}
\end{lstlisting}
\end{miniexammar}

\subsection{放置floats}
月曾经使用 Microsoft Word,它可以将floats放在用户想要放置的任何位置。 自从她转向 \LaTeX 一段时间以来,一切都很好,但现在月出现了问题。 一个图从输出中“消失了”。 一次次检查她的代码和输出后,她意外地发现下一页出现了这个图。 这真的让她很困惑。 由于 \TeX 的内部算法,技术上不可能将每个float排列在用户想要放置它们的位置。 根据 \textit{the \LaTeX{} Companion},
\begin{quotation}
``Floats are often problematic in the present version of \LaTeX, because the system was developed at a time when documents contained considerably less graphical material than they do today.''
\end{quotation}

然而\LaTeX{} 确实提供了一些选项,允许Yue 在某种程度上控制floats的位置。 对于figure或table环境,Yue 能够向它传递一个可选参数,指定所需的位置。 有五个放置说明符,它们可以按任何顺序组合在一起。
\begin{description}
\item[!] 忽略一些 \LaTeX{} 限制\footnote{\LaTeX{} 在尝试放置floats时有一些限制。 例如,如果float的高度大于页面高度的某种程度,则在尝试放置此float时无法将其放置在页面底部。}。
\item[h] 尝试将float准确地放置在环境被给定时的位置。 如果尝试失败并且除此之外没有其他说明符除了!被给定,说明符将更改为 t。
\item[t] 尝试将float放置在页面顶部。
\item[b] 尝试将float放置在页面底部。
\item[p] 尝试将浮动放置在float页面(由 \LaTeX{} 生成的用于放置floats的页面)
\end{description}
\LaTeX{} 尝试按照上述列表从上到下的顺序根据说明符放置一个float。 通常,文档的所有floats都可以被正确处理。 但是如果一个float被证明无法处理,作者应该调整(大概率减少)它的宽度和高度。

\subsection{Floats的目录}
正如开头提到的,在月的材料中,有很多floats。 她想知道是否有办法为他们提供快速参考。

与\verb=\tableofcontents= 一样,\LaTeX{} 提供了以下两个命令,分别打印文档中使用的所有图形列表和所有表格列表。
\begin{verbatim}
\listoffigures and \listoftables
\end{verbatim}
出现在列表中的float名称由float的caption定义。 如果caption看起来太长,Yue 可以向caption传递一个可选参数,该参数将显示在列表中。 另外,不要忘记编译文件至少两次以使列表正确显示。

\subsection{关于图的一些建议}
月被建议对图像使用矢量图,因为当图像与输出文件一起缩放时矢量图是无损的。

有几个软件包可以直接在 \LaTeX{} 中绘制图像,但使用它们都需要付出很大的努力。 建议使用现代工具生成适当的图像(例如 Mathematica 能够导出绘制的数学图。)。

\section{Tutorial 4: The Story of ZiYou and Abigail} % ZiYou for 子由
ZiYou and Abigail are the team leaders of the material about Calculus for the final exam. As team leaders, they have to deal with more problems than their colleagues. In this section, we will know about how ZiYou and Abigail manage to solve these problems and how their affection of each other grows.

\subsection{Managing Notes In \LaTeX}
In a review to the work of one of the team members, ZiYou finds several places that may not be clear enough for the readers. He decides to add some description to these unclear texts. These descriptions should not defer the reading of the main text, so ZiYou adjudicates on making them \emph{notes}. Two general ways for adding notes in \LaTeX{} are using \emph{footnotes} and using \emph{marginpars}.

A footnote in \LaTeX{} provides an annotation for a piece of text in the footer of the current page, and generates a number of the note which will appear as the superscript of the text being annotated. To use footnotes, ZiYou uses the command \verb=\footnote=.
\begin{miniexammar}{.55\textandmarginlen}{
This is something unclear\footnote{This means that ...}. And some other texts are here.
}
\begin{lstlisting}
This is something unclear\footnote{This means that ...}. And some other texts are here.
\end{lstlisting}
\end{miniexammar}

Different from ZiYou, Abigail prefers to use marginpars for annotation. A marginpar appears in the margin of the current page, but does not possess a number like a footnote does.
\begin{miniexammar}{.6\textandmarginlen}{
% We have to fake a margin par here.
\parbox{.68\textwidth}{This is something unclear. And some other texts are here.} \hspace{.03\textwidth}
\parbox{.27\textwidth}{That is, we have to ...}
}
\begin{lstlisting}
This is something unclear\marginpar{That is, we have to ...}. And some other texts are here.
\end{lstlisting}
\end{miniexammar}

A marginpar appears in the margin, and is at the same height as the text where the marginpar is given. When Abigail sees that ZiYou uses footnotes rather than marginpars, she asks him to change them because she thinks that marginpars are better. Certainly ZiYou doesn't agree with her, but he confers with what she claims, that the noting styles should agree with each other in a document.

To give a resolution about what noting style is to be used, ZiYou suggests that they play Tic Tac Toe, and Abigail thinks this is a good idea. After a few minutes, Abigail narrowly wins the game. ZiYou jokes that maybe she should let him win the next time, whereupon Abigail smilingly replies, ``That remains to be seen''. Nevertheless, she has a dim feeling that ZiYou deliberately lets her be the winner, but can't prove it from ZiYou's regretful expression.

\subsection{Merging Works Of The Team}
ZiYou and Abigail only need to collect the chapter.tex from the team members. They rename the files according to each person in a way that they can easily identify who is responsible for each file. After that, they input each file into the chapter.tex of an empty template. The final output can then be generated from the encapsulation.tex of the template.

Directly copying the contents is not a good option for inputting the files. ZiYou is about to search this on the Internet when Abigail discovers in the encapsulation.tex that the chapter.tex is directed into this file by the command \verb=\input=. The following code shows what is written in encapsulation.tex.
\begin{lstlisting}
\input{chapter.tex}
\end{lstlisting}
The argument passed to this command is the relative path of the target file. Abigail doesn't know what the term relative path means, so ZiYou, who has certain knowledge in computer science, explains to her that, the relative directory is the path of a file relative to the file in which the path is used. In this example, the file in which the relative directory is used is encapsulation.tex, and as a consequence of the two files being in the same folder, the relative directory of the chapter.tex is simply its name. If ZiYou and Abigail decide to put the files of the team members into a folder named Files for organization, then the directory should contain the folder's name plus a \verb=/= or \verb=\=, depending on the file system, at the beginning.

Not until they have completed the inputting had ZiYou had a glimpse at the search result on the Internet, where he finds another command, \verb=\include=. After seeing this webpage in detail, he then tells Abigail that \verb=\include= is a better choice here, for it somehow improves the compilation speed. Abigail has no clue about what compilation is and isn't interested in such technical stuffs, but she trusts ZiYou. She also kind of likes it when ZiYou, patiently and tenderly, explains what she doesn't understand to her. So she pretends to be curious and asks ZiYou to explain compilation to her.

So the final form of their chapters.tex is like this:
\begin{lstlisting}
\include{Files/The first file}

\include{Files/The second file}

\include{Files/The third file}
...
\end{lstlisting}
Note that the file extension (.tex) is not allowed to be used in \verb=\include=, while it can be used in \verb=\input=. Also, they have to make sure that the team members have not taken advantage of \verb=\include=, as it cannot be used in a file that is included by another. Fortunately, they can ascertain it since none of them knows about the command.

\subsection{Background, Headers, and Footers}
ZiYou and Abigail notice that the template for materials automatically adds the background, and the header and footer for each page. The encapsulation.tex loads the package background for background, and the package fancyhdr for headers and footers.

The headers and footers are set in the encapsulation.tex by the following code: (A line begins with a \verb=%= is \emph{commented}, so it will have no effect on the output .pdf file.)
\begin{lstlisting}
% Define the header and footer for pages.

% Place the number of the current page.
\fancyhead[LEH,ROH]{\bfseries\thepage}

% Beautify the display of chapter and section marks.
\renewcommand{\chaptermark}[1]{%
\markboth{#1}{}}
\renewcommand{\sectionmark}[1]{%
\markright{\thesection\ #1}}

\fancyhead[LOH]{\bfseries\rightmark}
\fancyhead[REH]{\bfseries\leftmark}

% Add copyright in the footer
\fancyfoot[COF,CEF]{\bfseries\copyright{} The XJTLU Math Club -- All rights reserved}
\end{lstlisting}
The header is controlled by the command \verb=\fancyhead= and the footer is controlled by the command \verb=\fancyfoot=. By inspecting the optional arguments, ZiYou makes a guess that L and R represent left and right, E and O represent even and odd (page number), and H and F represent header and footer, respectively. A look into the documentation of fancyhdr confirms this. He asks Abigail if she likes the page style. Abigail thinks that the author has a good taste, so they decide not to change this.

The background is set to the logo of math club. In fact, adding this graph somehow makes the document ugly, and even I didn't understand why this must be added to all materials. The head of the department at that time told me that this is a defense for those who use the materials in a prohibited way.

I really hope that this environment can improve to a state that even the materials are distributed by the most free licenses, there will be no one to steal our intelligent property.

\subsection{Index And Bibliography}
ZiYou and Abigail would like to listen to the readers' opinions about the previous materials so that they can refine the coming one according to them. A few number of the readers pointed out that it took them a lot of efforts to find specific terms in the materials and they would appreciate it if a list of important terms is added.

Abigail recalls that once when she tried to find something in a calculus textbook's appendix, she flipped the pages too much and turned to a section called Index, where terms are listed according to their pages. So ZiYou makes a search on the Internet and found that Indexing in \LaTeX{} can be easily done by a few commands.

First, for any important terms that they want to list in the Index page, they mark it with the command \verb=\index=. To print the Index page, the command \verb=\printindex= needs to be called. In the encapsulation.tex, simply uncomment the relating lines of code to do that. Finally, for the Index page to be printed, they have to, first, run \LaTeX{} once on encapsulation.tex, then, run MakeIndex once on the file, and finally, run \LaTeX{} twice on the file. The following example show the result of using Index.
\begin{miniexammar}{.35\textandmarginlen}{
\section*{Index}
limit, vi, 3\\
derivative, 3-5
}
\begin{lstlisting}
% At page vi
\index{limit}
% At page 3
\index{limit}
\index{derivative}
% At page 4
\index{derivative}
% At page 5
\index{derivative}
\end{lstlisting}
\end{miniexammar}

ZiYou makes definite integral and indefinite integral two separate index entries, while Abigail argues that they should be under the same term integral. To use subterms, the following syntax should be applied.
\begin{miniexammar}{.4\textandmarginlen}{
\section*{Index}
integral, 5
\par \hspace{2em} definite integral, 11
\par \hspace{2em} indefinite integral, 7
}
\begin{lstlisting}
% At page 5
\index{integral}
% At page 7
\index{integral!definite integral}
% At page 11
\index{integral!indefinite integral}
\end{lstlisting}
\end{miniexammar}

Abigail remembers that for the sake of academic integrity, they should add reference for each work of other people. \textsc{bib}\TeX{} is a good tool handling references (bibliographies). To use this tool, they have to prepare a \textsc{bib}\TeX{} database. This is a file with the extension .bib, in which entries are contained. Filling in the entries is a tedious work, but fortunately most academic websites provides the facility to allow an user to directly download a .bib file for the source he wants to use. The following code shows a typical database entry:
\begin{lstlisting}
@article{may1979alpha,
	title="Alpha-particle-induced soft errors in dynamic memories",
	author="T.C. {May} and M.H. {Woods}",
	journal="IEEE Transactions on Electron Devices",
...
}
\end{lstlisting}
ZiYou and Abigail need not to worry about the details in the entry. The only thing they need to remember is the label of the entry, the one immediately after the \verb={=, because it is to be used in the \verb=\cite= command to produce a reference to that source. Each source that is referenced in the document appears in the Bibliography page, which is controlled by the two following commands:
\begin{lstlisting}
\bibliography{file-list}
\bibliographystyle{style}
\end{lstlisting}
, where file-list is a list of database files, and style is the bibliography style according to which the bibliography is to be printed. \url{https://www.overleaf.com/learn/latex/Bibtex_bibliography_styles} shows all predefined \textsc{bib}\TeX{} styles.

\part{Internal Documentation}
This part is the internal documentation for xjtlumath. It is automatically generated by the tool doc. This part is for the maintenance of the package and is not recommended to be read by the general users of the package unless they want to know the implementation details about the package.
\DocInput{xjtlumath.dtx} 

\clearpage
\DocInput{mydoc.dtx}
\end{document}