\section{教程 4: 子由和 Abigail 的故事} % ZiYou for 子由
子由和Abigail是期末考微积分教材的组长。 作为团队领导,他们要处理的问题比同事多。 在本节中,我们将了解子由和Abigail是如何解决这些问题的,以及他们之间的感情是如何增长的。

\subsection{\LaTeX 里的注解管理}
在对其中一位团队成员的工作进行审查时,子由发现了几个读者可能不太清楚的地方。 他决定为这些不清楚的文本添加一些描述。 这些描述不应耽误正文的阅读,因此子由裁定将它们制作为\emph{注解}。 在 \LaTeX{} 中添加注释的两种常用方法是使用 \emph{footnotes} 和使用 \emph{marginpars}。

\LaTeX{} 中的脚注在当前页面页脚中为一段文本提供注释,并生成一个计数器,这计数器将显示为被注释文本的上标。 要使用脚注,子由使用命令\verb=\footnote=。
\begin{miniexammar}{.55\textandmarginlen}{
This is something unclear\footnote{This means that ...}. And some other texts are here.
}
\begin{lstlisting}
This is something unclear\footnote{This means that ...}. And some other texts are here.
\end{lstlisting}
\end{miniexammar}

与子由不同,Abigail 更喜欢使用 marginpars 进行标注。 marginpar 出现在当前页面的页边空白处,但不像脚注那样具有数字。
\begin{miniexammar}{.6\textandmarginlen}{
% We have to fake a margin par here.
\parbox{.68\textwidth}{This is something unclear. And some other texts are here.} \hspace{.03\textwidth}
\parbox{.27\textwidth}{That is, we have to ...}
}
\begin{lstlisting}
This is something unclear\marginpar{That is, we have to ...}. And some other texts are here.
\end{lstlisting}
\end{miniexammar}

marginpar 出现在页边中,并且与给出 marginpar 的文本高度相同。 当Abigail看到子由使用脚注而不是marginpar时,她请求他更改它们,因为她认为marginpar更好。子由当然不同意她的说法,但他同意她的另一个说法,即在一个文件中注解的风格应该相互一致。

为了确定使用什么注记方式,子由建议他们玩井字游戏,Abigail认为这是个好主意。 几分钟后,Abigail以微弱优势赢得了比赛。 子由打趣说下次或许她应该让他赢,于是Abigail笑着回答:“那还有待观察”。 尽管如此,她还是有一种隐隐的感觉,就是子由是故意让她成为赢家的,但从子由的遗憾表情上却无法证明。

\subsection{合并组内工作}
子由和Abigail只需要从团队成员那里收集chapter.tex。 他们根据每个人重命名文件,以便他们可以轻松识别每个文件的负责人。 之后,他们将每个文件输入到空模板的 Chapter.tex 中。 然后可以从模板的 encapsulation.tex 生成最终输出。

直接复制内容不是输入文件的好选择。 子由正要在网上搜索这个,Abigail在encapsulation.tex中发现chapter.tex是通过命令\verb=\input=被导入到这个文件的。 以下代码显示了 encapsulation.tex 中的内容。 
\begin{lstlisting}
\input{chapter.tex}
\end{lstlisting}
传递给该命令的参数是目标文件的相对路径。Abigail不知道相对路径是什么意思,所以对计算机有一定了解的子由给她解释说,相对目录就是文件相对于使用该路径的文件的路径。在这个例子中,使用相对目录的文件是encapsulation.tex,由于两个文件在同一个文件夹中,chapter.tex的相对目录就是它的名字。如果子由和Abigail决定将团队成员的文件放入名为 Files 的文件夹中,则该目录应位于开始包含文件夹名称加上 \verb=/= 或 \verb=\=,具体取决于文件系统。由于\LaTeX{}中\verb=\=是保留字,所以应使用\verb=/=.

直到他们输入完毕,子由才在网上看到了搜索结果,发现了另一个命令,\verb=\include=。详细看了这个网页后,他告诉Abigail,\verb=\include= 是这里更好的选择,因为它在某种程度上提高了编译速度。Abigail不知道什么是编译,对这种技术性的东西也不感兴趣,但她信任子由。她也有点喜欢子由,当他耐心而温柔地向她解释她不明白的事情。于是她装作好奇,让子由给她解释编译。

所以他们的 Chapters.tex 最终的形式是这样的:
\begin{lstlisting}
\include{Files/The first file}

\include{Files/The second file}

\include{Files/The third file}
...
\end{lstlisting}
请注意,文件扩展名 (.tex) 不允许在 \verb=\include= 中使用,而可以在 \verb=\input= 中使用。 此外,他们必须确保团队成员没有利用 \verb=\include=,因为它不能在另一个被\verb=\include=包含的文件中使用。 幸运的是,他们可以确定这一点,因为团队成员们都不知道此命令。

\subsection{背景,页眉和页脚}
子由和Abigail注意到素材模板会自动为每个页面添加背景、页眉和页脚。 encapsulation.tex 为背景加载包背景,为页眉和页脚加载包 fancyhdr。

页眉和页脚通过以下代码在 encapsulation.tex 中设置:(以 \verb=%= 开头的行是 \emph{被注释的},因此它不会对输出的 .pdf 文件产生影响。)
\begin{lstlisting}
% Define the header and footer for pages.

% Place the number of the current page.
\fancyhead[LEH,ROH]{\bfseries\thepage}

% Beautify the display of chapter and section marks.
\renewcommand{\chaptermark}[1]{%
\markboth{#1}{}}
\renewcommand{\sectionmark}[1]{%
\markright{\thesection\ #1}}

\fancyhead[LOH]{\bfseries\rightmark}
\fancyhead[REH]{\bfseries\leftmark}

% Add copyright in the footer
\fancyfoot[COF,CEF]{\bfseries\copyright{} The XJTLU Math Club -- All rights reserved}
\end{lstlisting}
页眉由命令\verb=\fancyhead= 控制,页脚由命令\verb=\fancyfoot= 控制。 通过检查可选参数,子由猜测 L 和 R 代表左和右,E 和 O 代表偶数和奇数(页码),H 和 F 分别代表页眉和页脚。 看一下 fancyhdr 的文档就证实了这一点。 他问Abigail是否喜欢页面样式。 Abigail认为作者的品味很好,所以他们决定不改变这一点。

背景设置为数学社的标志。 事实上,添加这个图不知何故使文档变得丑陋,甚至我也不明白为什么必须将其添加到所有材料中。 当时的部门负责人告诉我,这是对那些以被禁止的方式使用这些材料的人的一种防御。

我真的希望这个环境能够改善到一种状态,即使是以最自由的许可证分发材料,也不会有人窃取我们的知识产权。

\subsection{索引和参考文献}
子由和Abigail想听听读者对之前材料的意见,以便他们根据他们的意见来完善未来的材料。 一些读者指出,他们花了很多精力在材料中找到特定术语,如果添加了重要术语列表,他们将不胜感激。

Abigail回忆说,有一次,当她试图在微积分教科书的附录中找到一些东西时,她翻了太多页,然后转向了一个名为“索引”的部分,其中的术语是根据它们的页数列出的。 于是子由在网上一搜,发现\LaTeX{}中的Indexing,几条命令就可以轻松搞定。

首先,对于他们想要在索引页面中列出的任何重要术语,他们使用命令\verb=\index= 来标记它。 要打印索引页,需要调用命令 \verb=\printindex=,并且在文档环境之前应该调用命令 \verb=\makeindex=。 在 encapsulation.tex 中,只需取消注释相关代码行即可。 最后,要打印索引页,他们必须首先在 encapsulation.tex 上运行 \LaTeX{} 一次,然后在文件上运行 MakeIndex 一次,最后在文件上运行 \LaTeX{} 两次。 以下示例显示了使用 Index 的结果。
\begin{miniexammar}{.35\textandmarginlen}{
\section*{Index}
limit, vi, 3\\
derivative, 3-5
}
\begin{lstlisting}
% At page vi
\index{limit}
% At page 3
\index{limit}
\index{derivative}
% At page 4
\index{derivative}
% At page 5
\index{derivative}
\end{lstlisting}
\end{miniexammar}

子由把定积分和不定积分列为两个单独的索引条目,而Abigail则认为它们应该在同一个索引:积分下。 要使用子索引项,应应用以下语法。
\begin{miniexammar}{.4\textandmarginlen}{
\section*{Index}
integral, 5
\par \hspace{2em} definite integral, 11
\par \hspace{2em} indefinite integral, 7
}
\begin{lstlisting}
% At page 5
\index{integral}
% At page 7
\index{integral!definite integral}
% At page 11
\index{integral!indefinite integral}
\end{lstlisting}
\end{miniexammar}

Abigail记得,为了学术诚信,他们应该为其他人的每部作品添加参考。 B\textsc{ib}\TeX{} 是一个很好的工具处理参考(参考书目)。 要使用这个工具,他们必须准备一个 B\textsc{ib}\TeX{} 数据库。 这是一个扩展名为 .bib 的文件,其中包含条目。 填写条目是一项乏味的工作,但幸运的是,大多数学术网站都提供了允许用户直接下载他想要使用的源的 .bib 文件的工具。 以下代码显示了一个典型的数据库条目:
\begin{lstlisting}
@article{may1979alpha,
	title="Alpha-particle-induced soft errors in dynamic memories",
	author="T.C. {May} and M.H. {Woods}",
	journal="IEEE Transactions on Electron Devices",
...
}
\end{lstlisting}
子游和阿比盖尔不用担心条目中的细节。 他们唯一需要记住的是条目的标签,紧跟在 \verb={= 之后的标签,因为它会在 \verb=\cite= 命令中使用以产生对该来源的引用。 文档中引用的每个来源都出现在参考书目页面中,该页面由以下两个命令控制:
\begin{lstlisting}
\bibliography{file-list}
\bibliographystyle{style}
\end{lstlisting}
,其中 file-list 是数据库文件的列表,而 style 是参考书目打印的书目样式。 \url{https://www.overleaf.com/learn/latex/Bibtex_bibliography_styles} 显示所有预定义的 B\textsc{ib}\TeX{} 样式。

\subsection{双边打印}
现在子由和Abigail已经完成了单独的文件的合并,正在计划完成工作后休息一下,但还有更多的事情等着他们。

默认情况下,当文档类为 book 时,\LaTeX{} 使用双边打印。起初子由和Abigail没有注意到这一点(他们不知道),直到Abigail看到输出的页面没有对齐。也就是说,在一个页面上,内容向左伸出,而在下一页,内容向右伸出。Abigail不想再给子由添麻烦了,她自己搜了一下,发现这就是所谓的双边,专用于书籍等。

Abigail试图通过打开一本书并检查其布局来理解这一点。她注意到里面的一部分\footnote{当打开一本书时,出现在右侧的第一页编号为 1,因此出现在右侧的所有奇数页都是奇数。} 页面是粘在一起使它们形成一本书。她猜测,双边在每页靠里的地方留下了额外的空间。然而,结果却与她的直觉相矛盾。在双面模式下,\LaTeX{} 会缩小每个页面的内部部分。这是因为 \LaTeX{} 想要为位于外部部分的页边提供更多空间。

子由从Abigail那里了解到了这一点,所以他们开始了另一次审查。他们很快就发现了问题。对于目录和其他一些页面(如前言),模板使用罗马编号。对于正文,页码更改为阿拉伯数字。然而,页面的布局并不取决于它是出现在左侧还是右侧。相反,它完全由页数决定。比如说,罗马数字以iii结尾,那么当阿拉伯数字从1开始时,左边会出现相应的页面,这是一场灾难。为了解决这个问题,子由找到了为双边模式设计的命令\verb=\cleardoublepage=。它通过(可能)添加新页面来确保命令之后的内容将出现在右侧的页面上。当然,新页面将是空的。但Abigail认为读者可能会将其误解为印刷失败的迹象。为了明确地识别这种行为,在 xjtlumath(准确地说,是模板加载的子包 xjtlumaterial)中,该命令被更新,在该命令添加的每个空页面上添加句子“This page is intentionally left blank”。

\subsection{他们故事的一个结束}
子由和Abigail关于材料写作的故事到此结束,但正如一句老话所说的“结束也是开始”,他们其他的故事才刚刚开始。 Abigail钦佩子由在计算机科学方面的广泛知识,并在子由教她时享受它。 子由也受到了Abigail的能量的启发,尤其是在工作中。 能找到像Abigail这样有趣又美丽的女孩,他感到很幸运,她的幽默让子由笑了好几次。

于是两人在浪漫的餐厅订了位子来庆祝他们的合作,而我一个人坐在宿舍里,努力完成这份文件。 不过,我希望你们,未来的数学俱乐部成员,不仅可以找到知识和经验,还可以像子由和Abigail一样找到爱,只是不是在虚构的故事中。